\chapter{PENDAHULUAN}

\section{Latar Belakang}

% Ubah paragraf-paragraf berikut sesuai dengan latar belakang dari tugas akhir
Perkembangan teknologi digital telah mengubah prospek industri musik secara dramatis dalam beberapa dekade terakhir. 
Dari era penjualan album fisik dan merchandise konvensional, industri musik telah berevolusi memasuki era streaming 
serta pengalaman digital yang lebih imersif. Ruddinet al. (2022) menyimpulkan bahwa digitalisasi dapat membawa banyak
peluang untuk memperluas pasar dan meraup lebih banyak keuntungan, perusahaan dapat memotong anggaran untuk memproduksi
barang-barang fisik seperti kaset, piringan hitam, dan CD dan beralih ke platform-platform digital [1].
\\
Merchandise pada umumnya seperti album, lightstick, photocard, dan berbagai pernak-pernik fisik telah lama menjadi bagian
komprehensif dari pengalaman penggemar musik, terutama dalam kultur penggemar K-pop dan musik pop global. Nurjanah dan
Ikhsan (2020) mencatat bahwa penggemar K-Pop sering mengoleksi berbagai jenis merchandise, termasuk lightstick, kaos,
gantungan kunci, topi, pin, stiker, dan album musik [2]. Oliver (2020) menambahkan bahwa album musik K-Pop menjadi 
salah satu merchandise terlaris karena menawarkan lebih dari sekadar CD, termasuk kemasan mewah, desain dengan berbagai
konsep, serta konten menarik seperti album foto, kartu foto, dan poster [3]. Namun, merchandise fisik ini memiliki 
keterbatasan dalam hal interaktivitas, personalisasi, dan nilai jangka panjang bagi penggemar. Seiring dengan perkembangan
zaman industri musik Korea telah mengalami transformasi digital yang signifikan. Parc dan Kim (2020) membahas 
munculnya album digital atau Kihno/KIT album K-pop, yang memungkinkan penggemar untuk mendengarkan musik langsung melalui
perangkat smartphone, menawarkan efisiensi yang lebih tinggi dibandingkan dengan album fisik [4]. Arassy et al. (2021)
melaporkan bahwa dalam sebuah survey sebanyak 141 responden menghasilkan, 69,5% menyatakan melakukan pembelian terhadap
album fisik, 50,4% membeli album digital Kihno/KIT, dan 81,6% melakukan pembelian terhadap merchandise K-pop [5],
menunjukkan bahwa baik format digital maupun fisik tetap relevan bagi penggemar. Di sisi lain, perkembangan teknologi
blockchain dan konsep metaverse membuka peluang baru untuk menciptakan merchandise digital yang lebih dinamis, interaktif,
dan bernilai.


\section{Rumusan Masalah}

% Ubah paragraf berikut sesuai dengan rumusan masalah dari tugas akhir
Berdasarkan hal yang telah dipaparkan di latar belakang, \lipsum[4]

\section{Batasan Masalah atau Ruang Lingkup}

\lipsum[6]

\section{Tujuan}

% Ubah paragraf berikut sesuai dengan tujuan penelitian dari tugas akhir
Tujuan dari penelitian ini adalah \lipsum[7][1-14]

\section{Manfaat}

% Ubah paragraf berikut sesuai dengan tujuan penelitian dari tugas akhir
Manfaat dari penelitian ini adalah \lipsum[8][1-14]

\chapter{PENDAHULUAN}

\section{Latar Belakang}

% Ubah paragraf-paragraf berikut sesuai dengan latar belakang dari tugas akhir
Perkembangan teknologi digital telah mengubah prospek industri musik secara dramatis dalam beberapa dekade terakhir. 
Mulai dari era penjualan piringan hitam ke kaset lalu dilanjutkan dengan pemakaian CD, industri musik telah berevolusi 
memasuki era streaming serta pengalaman digital yang lebih imersif. Ruddinet al. (2022) menyimpulkan bahwa digitalisasi 
dapat membawa banyak peluang untuk memperluas pasar dan meraup lebih banyak keuntungan seperti, memotong anggaran untuk 
memproduksi barang-barang fisik seperti kaset, piringan hitam, dan CD dan beralih ke platform-platform digital \parencite{Ruddin2022}. Adanya
perkembangan teknologi yang sangat pesat memungkinkan produsen musik memasarkan beberapa karya terbaiknya secara virtual
tanpa menggunakan subject musisi nyata. Salah satu contoh nyata dari evolusi ini adalah munculnya MAVE, sebuah grup K-pop
virtual yang diciptakan menggunakan Unreal Engine dan MetaHuman. MAVE mendemonstrasikan bagaimana teknologi canggih dapat
digunakan untuk menciptakan musisi virtual yang sangat realistis dan menarik \parencite{EpicGames2023}. Kehadiran grup seperti MAVE tidak hanya
membuka peluang baru dalam produksi musik, tetapi juga dalam pengelolaan merchandise. Industri musik kini dapat 
mengeksplorasi bisnis baru dalam menciptakan dan memasarkan merchandise digital yang unik dan interaktif, melampaui 
batasan merchandise fisik pada umumnya.
\\

Merchandise berbentuk fisik seperti album, lightstick, photocard, dan berbagai pernak-pernik fisik lainnya telah lama menjadi bagian
komprehensif dari pengalaman penggemar musik, ter- utama dalam kultur penggemar K-pop dan musik pop global. Nurjanah dan
Ikhsan (2020) mencatat bahwa penggemar K-Pop sering mengoleksi berbagai jenis merchandise, termasuk lightstick, kaos,
gantungan kunci, topi, pin, stiker, dan album musik \parencite{Nurjanah2020}. Oliver (2020) menambahkan bahwa album musik K-Pop menjadi 
salah satu merchandise terlaris karena menawarkan lebih dari sekadar CD, termasuk kemasan mewah, desain dengan berbagai
konsep, serta konten menarik seperti album foto, kartu foto, dan poster \parencite{Oliver2020}. Namun, merchandise fisik ini memiliki 
keterbatasan dalam hal interaktivitas, personalisasi, dan nilai jangka panjang bagi penggemar. Arassy et al. (2021)
melaporkan bahwa dalam sebuah survey sebanyak 141 responden menghasilkan, 69,5\% menyatakan melakukan pembelian terhadap
album fisik, 50,4\% membeli album digital Kihno/KIT, dan 81,6\% melakukan pembelian terhadap merchandise K-pop \parencite{Saputera2024},
hal ini menunjukkan bahwa baik format digital maupun fisik tetap relevan bagi penggemar. Di sisi lain, perkembangan teknologi
blockchain dan konsep metaverse membuka peluang baru untuk menciptakan merchandise digital yang lebih dinamis, interaktif, dan
bernilai.
\\

Integrasi teknologi blockchain, khususnya melalui penggunaan Non-Fungible Tokens (NFT), membuka peluang untuk menciptakan 
merchandise digital yang unik, dapat diverifikasi keasliannya, dan memiliki nilai koleksi yang potensial. Saputera et al. 
(2024) mengemukakan bahwa dalam dunia metaverse, musisi dan produser musik memiliki peluang untuk membentuk grup musik atau
band NFT fiksi virtual. Mereka dapat mendistribusikan NFT musik, menjual tiket NFT, dan menjual merchandise band dalam 
dunia metaverse parencite{Saputera2024}. NFT dapat berfungsi sebagai \"tiket digital\" yang tidak hanya memberikan akses ke konser fisik, 
tetapi juga membuka pintu ke pengalaman virtual eksklusif dalam metaverse. Untuk merealisasikan konsep merchandise digital
berbasis metaverse yang interaktif dan imersif, diperlukan teknologi canggih yang mampu menciptakan lingkungan virtual 
serta avatar dari seorang musisi asli di kehidupan nyata. Salah satu platform yang menawarkan kemampuan tersebut adalah 
Unreal Engine 5.
\\

Unreal Engine 5, sebagai platform pengembangan utama yang telah menyediakan fondasi yang cukup kuat untuk menciptakan
lingkungan virtual yang sangat detail dan responsif. Kemampuan rendering secara real-time serta kualitas visual yang
tinggi maka sangat tepat, Unreal Engine 5 dijadikan tools untuk penciptaan lingkungan konser virtual yang sangat 
realistis ditambah dengan pencahayaan dinamis dan efek partikel yang mengesankan. Salah satu fitur revolusioner yang
diintegrasikan dalam Unreal Engine 5 adalah MetaHuman Creator. Teknologi ini memungkinkan pembuatan avatar digital 
hyperrealistic dengan tingkat detail yang luar biasa. MetaHuman Creator mempercepat proses pembuatan karakter digital
yang biasanya memakan waktu berbulan-bulan menjadi hanya beberapa jam [8]. Dengan teknologi ini, pengembang dapat 
menciptakan representasi digital yang sangat akurat dari musisi sehingga memungkinkan penggemar untuk saling  
berinteraksi dengan berbagai macam fitur yang akan disediakan.
\\

Untuk meningkatkan realisme dan keotentikan avatar digital, teknologi Facial Capture dapat diintegrasikan ke dalam
proses pembuatan karakter. Facial Capture memungkinkan perekaman dan reproduksi ekspresi wajah artis dengan akurasi
tinggi. Teknologi ini dapat menangkap nuansa ekspresi, gerakan mulut yang halus serta menampilkan emosi dan lip-sync
yang realistis selama pertunjukan virtual \parencite{Zeng2020}. Integrasi teknologi-teknologi ini memungkinkan penciptaan Digital
Twin dari artis musik \- representasi virtual yang tidak hanya mirip secara visual, tetapi juga mampu mereplikasi
gerakan dan ekspresi dengan tingkat akurasi yang tinggi. Konsep Digital Twin ini membuka peluang untuk pengalaman
konser yang dapat diakses kapan saja dan di mana saja, melampaui batasan waktu dan ruang. Dalam konteks merchandise digital
berbasis NFT, Digital Twin ini dapat dimanfaatkan untuk menciptakan pengalaman konser virtual yang unik. Penggemar yang
membeli tiket konser dalam bentuk NFT tidak hanya mendapatkan akses ke konser fisik, tetapi juga memperoleh hak untuk
menikmati pengalaman konser virtual dari perspektif kursi sesuai dengan tiket yang mereka beli. Ini menciptakan pengalaman
Point of View (POV) yang personal dan eksklusif dalam lingkungan metaverse.



\section{Rumusan Masalah}

% Ubah paragraf berikut sesuai dengan rumusan masalah dari tugas akhir

Penelitian ini bertujuan untuk menyelidiki proses perancangan dan pengembangan avatar hyperrealistic untuk Digital
Twin konser musik menggunakan teknologi terkini seperti Unreal Engine 5 dan MetaHuman Creator. Dalam konteks ini,
pertanyaan utama yang muncul adalah bagaimana mengintegrasikan teknologi Facial Capture untuk meningkatkan realisme
ekspresi wajah dan lip-sync pada avatar musisi, sehingga dapat menciptakan representasi virtual yang sangat mirip
dengan artis aslinya. Selanjutnya, penelitian ini juga akan mengeksplorasi metode untuk mengoptimalkan performa avatar
dalam lingkungan metaverse berbasis Web 3.0, dengan tujuan mendukung pengalaman konser virtual yang interaktif dan imersif. 
Tantangan utama yang dihadapi adalah bagaimana menyeimbangkan kualitas visual yang tinggi dengan kebutuhan performa real-time 
dalam konteks streaming web, sambil tetap mempertahankan tingkat interaktivitas yang diperlukan untuk menciptakan pengalaman 
konser virtual yang menarik dan autentik.

\section{Batasan Masalah atau Ruang Lingkup}

Peniliti membatasi masalah untuk membantu mencapai tujuan sebagai berikut.

\begin{enumerate}
    \item Pengembangan avatar dilakukan menggunakan Unreal Engine 5 dan MetaHuman Creator.
    \item Integrasi Facial Capture terbatas pada ekspresi wajah dan lip-sync, tidak termasuk motion capture untuk gerakan tubuh penuh.
    \item Implementasi metaverse dibatasi pada lingkungan berbasis Web 3.0 yang dapat diakses melalui browser web.
    \item Aspek blockchain dan NFT tidak termasuk dalam lingkup pengembangan avatar, namun akan dipertimbangkan dalam integrasi sistem secara keseluruhan.
\end{enumerate}

% \begin{enumarate}
% \item Pengembangan avatar dilakukan menggunakan Unreal Engine 5 dan MetaHuman Creator.
% \item Integrasi Facial Capture terbatas pada ekspresi wajah dan lip-sync, tidak termasuk motion capture untuk gerakan tubuh penuh.
% \item Implementasi metaverse dibatasi pada lingkungan berbasis Web 3.0 yang dapat diakses melalui browser web.
% \item Aspek blockchain dan NFT tidak termasuk dalam lingkup pengembangan avatar, namun akan dipertimbangkan dalam integrasi sistem secara keseluruhan.
% \end{enumarate}

\section{Tujuan}

% Ubah paragraf berikut sesuai dengan tujuan penelitian dari tugas akhir

Penelitian ini bertujuan untuk merancang dan mengembangkan avatar hyperrealistic untuk Digital Twin konser musik dengan memanfaatkan
teknologi terkini seperti Unreal Engine 5 dan MetaHuman Creator. Dalam proses ini, penelitian akan mengintegrasikan teknologi Facial
Capture guna meningkatkan realisme ekspresi wajah dan lip-sync pada avatar musisi, sehingga menciptakan representasi digital yang 
lebih autentik dan menarik. Lebih lanjut, penelitian ini akan berfokus pada optimalisasi performa avatar dalam lingkungan metaverse 
berbasis Web 3.0, dengan tujuan mendukung pengalaman konser virtual yang interaktif dan imersif. Melalui upaya-upaya ini, penelitian 
bertujuan untuk menghasilkan prototype avatar Digital Twin yang tidak hanya memiliki kualitas visual yang tinggi, tetapi juga dapat 
diimplementasikan secara efektif sebagai bagian dari merchandise digital berbasis NFT untuk industri musik. Dengan demikian, 
penelitian ini berupaya untuk membuka jalan bagi inovasi dalam industri hiburan musik, menggabungkan teknologi cutting-edge dengan 
kreativitas untuk menciptakan pengalaman penggemar yang lebih kaya dan bernilai tambah.

\section{Manfaat}
Dengan adanya tugas akhir ini, penulis berharap dapat mendapatkan manfaat sebagai berikut.

\begin{enumerate}
    \item Menyediakan avatar hyperrealistic yang dapat digunakan sebagai representasi digital artis dalam berbagai konteks, termasuk konser virtual, meet and greet digital, dan promosi.
    \item Membuka peluang untuk merchandise digital baru berbasis avatar, meningkatkan engagement penggemar dan potensi pendapatan.
    \item Menawarkan cara baru untuk berinteraksi dengan penggemar melalui avatar digital yang sangat mirip dengan diri mereka.
    \item Meningkatkan pengalaman interaksi dengan artis favorit melalui avatar yang sangat realistis dalam lingkungan virtual.
    \item Membuka peluang eksplorasi integrasi avatar hyperrealistic dengan teknologi AI untuk menciptakan interaksi yang lebih dinamis dan personal.
\end{enumerate}

% Ubah paragraf berikut sesuai dengan tujuan penelitian dari tugas akhir
% \begin{enumarate}
%     \item Menyediakan avatar hyperrealistic yang dapat digunakan sebagai representasi digital artis dalam berbagai konteks, termasuk konser virtual, meet and greet digital, dan promosi.
%     \item Membuka peluang untuk merchandise digital baru berbasis avatar, meningkatkan engagement penggemar dan potensi pendapatan.
%     \item Menawarkan cara baru untuk berinteraksi dengan penggemar melalui avatar digital yang sangat mirip dengan diri mereka.
%     \item Meningkatkan pengalaman interaksi dengan artis favorit melalui avatar yang sangat realistis dalam lingkungan virtual.
%     \item Membuka peluang eksplorasi integrasi avatar hyperrealistic dengan teknologi AI untuk menciptakan interaksi yang lebih dinamis dan personal.
% \end{enumarate}

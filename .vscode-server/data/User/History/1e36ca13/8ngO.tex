% Pengaturan ukuran teks dan bentuk halaman dua sisi
\documentclass[12pt]{book}

% Pengaturan ukuran halaman dan margin
\usepackage[a4paper,top=30mm,left=30mm,right=20mm,bottom=25mm]{geometry}

% Pengaturan ukuran spasi
\usepackage[singlespacing]{setspace}

% Pengaturan caption untuk tabel
\usepackage{caption}

% Judul dokumen
\title{Proposal Tugas Akhir ITS}
\author{Musk, Muhammad Daffa' Fisabilillah}

% Pengaturan detail pada file PDF
\usepackage[pdfauthor={\@author},bookmarksnumbered,pdfborder={0 0 0}]{hyperref}


% Pengaturan ukuran indentasi
\setlength{\parindent}{2em}

% Package lainnya
\usepackage{changepage}
\usepackage{etoolbox} % Mengubah fungsi default

% Pengaturan jenis karakter
\usepackage[utf8]{inputenc}

\usepackage[style=ieee, backend=biber]{biblatex}
\usepackage{enumitem} % Pembuatan list
\usepackage{lipsum} % Pembuatan template kalimat
\usepackage{graphicx} % Input gambar
\usepackage{longtable} % Pembuatan tabel
\usepackage[table,xcdraw]{xcolor} % Pewarnaan tabel
\usepackage{eso-pic} % Untuk menggunakan background image di halaman
\usepackage{txfonts} % Font times
\usepackage{changepage} % Pembuatan teks kolom
\usepackage{multicol} % Pembuatan kolom ganda
\usepackage{multirow} % Pembuatan baris ganda
\usepackage{tabularx} % Untuk mengatur kolom, seperti grid pada CSS
\usepackage{wrapfig}
\usepackage{float}

% Pengaturan format daftar isi, daftar gambar, dan daftar tabel
\usepackage[titles]{tocloft}
\setlength{\cftsecindent}{2em}
\setlength{\cftsubsecindent}{2em}
\setlength{\cftbeforechapskip}{1.5ex}
\setlength{\cftbeforesecskip}{1.5ex}
\setlength{\cftbeforetoctitleskip}{0cm}
\setlength{\cftbeforeloftitleskip}{0cm}
\setlength{\cftbeforelottitleskip}{0cm}
\renewcommand{\cfttoctitlefont}{\hfill\Large\bfseries} % command untuk membuat heading bold dan besar
\renewcommand{\cftaftertoctitle}{\hfill}
\renewcommand{\cftloftitlefont}{\hfill\Large\bfseries}
\renewcommand{\cftafterloftitle}{\hfill}
\renewcommand{\cftlottitlefont}{\hfill\Large\bfseries}
\renewcommand{\cftafterlottitle}{\hfill}

% Definisi untuk "Hati ini sengaja dikosongkan"
\patchcmd{\cleardoublepage}{\hbox{}}{
  \thispagestyle{empty}
  \vspace*{\fill}
  \begin{center}\textit{[Halaman ini sengaja dikosongkan]}\end{center}
  \vfill}{}{}

  % Pengaturan penomoran halaman
\usepackage{fancyhdr}
\fancyhf{}
\renewcommand{\headrulewidth}{0pt}
\pagestyle{fancy}
\fancyfoot[C,CO]{\thepage}
\patchcmd{\chapter}{plain}{fancy}{}{}
\patchcmd{\chapter}{empty}{plain}{}{}

% Pengaturan format judul bab
\usepackage{titlesec}
\renewcommand{\thesection}{\thechapter.\arabic{section}}
\titleformat{\chapter}[hang]{\centering\bfseries\large}{BAB\ \arabic{chapter}\ }{0ex}{\vspace{0ex}\centering}
\titleformat*{\section}{\large\bfseries}
\titleformat*{\subsection}{\normalsize\bfseries}
\titlespacing{\chapter}{0ex}{0ex}{4ex}
\titlespacing{\section}{0ex}{1ex}{0ex}
\titlespacing{\subsection}{0ex}{0.5ex}{0ex}
\titlespacing{\subsubsection}{0ex}{0.5ex}{0ex}
\setcounter{secnumdepth}{3} % Untuk memberi penomoran pada \subsubsection

\counterwithin{figure}{chapter}
\counterwithin{table}{chapter}

% Mengganti figure dan table menjadi gambar dan tabel
\renewcommand{\figurename}{Gambar}
\renewcommand{\tablename}{Tabel}

\input{pustaka/tanda-hubung.tex}

% Menambahkan resource daftar pustaka
\addbibresource{pustaka/pustaka.bib}

% Isi keseluruhan dokumen
\begin{document}
  % Nomor halaman pembuka dimulai dari sini
  \pagenumbering{roman}

  % Atur ulang penomoran halaman
  \setcounter{page}{1}

  % Sampul Bahasa Indonesia
  \newcommand\covercontents{sampul/konten-id.tex}
  \input{sampul/sampul-luar.tex}

  % Lembar pengesahan
  \input{lainnya/lembar-pengesahan.tex}
  \cleardoublepage

  % Abstrak
  \input{lainnya/abstrak.tex}
  \cleardoublepage

  \input{lainnya/abstrak-en.tex}
  \cleardoublepage

  \begin{spacing}{1.5}
    % Daftar isi
    \renewcommand*\contentsname{DAFTAR ISI}
    \addcontentsline{toc}{chapter}{\contentsname}
    \tableofcontents
    \cleardoublepage

    % Daftar gambar
    \renewcommand*\listfigurename{DAFTAR GAMBAR}
    \addcontentsline{toc}{chapter}{\listfigurename}
    \listoffigures
    \cleardoublepage

    % Daftar tabel
    \renewcommand*\listtablename{DAFTAR TABEL}
    \addcontentsline{toc}{chapter}{\listtablename}
    \listoftables
    \cleardoublepage
  \end{spacing}

  % Nomor halaman isi dimulai dari sini
  \pagenumbering{arabic}

  % Konten pendahuluan
  \input{konten/1-pendahuluan.tex}
  \cleardoublepage

  % Konten tinjauan pustaka
  \input{konten/2-tinjauan-pustaka.tex}
  \cleardoublepage

  % Konten metodologi
  \input{konten/3-metodologi.tex}
  \cleardoublepage

  % Konten lainnya
  \input{konten/4-lainnya.tex}
  \cleardoublepage

  \input{konten/5-jadwal-penelitian.tex}
  \cleardoublepage

  % Daftar pustaka
  \chapter*{DAFTAR PUSTAKA}
  \addcontentsline{toc}{chapter}{DAFTAR PUSTAKA}
  \renewcommand\refname{}
  \vspace{2ex}
  \renewcommand{\bibname}{}
  \begingroup
    \def\chapter*#1{}
    \printbibliography
  \endgroup


\end{document}
